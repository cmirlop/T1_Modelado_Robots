\documentclass[11pt]{article}

%------------------------------------------------------------------
%------------------------------------------------------------------
%------------------------------------------------------------------
% Informació de l'informe

\newcommand{\titol}{
	 Instrucciones para la realización de informes técnicos y científicos
	 }

\newcommand{\titolcap}{Instrucciones para la realización de informes}

\newcommand{\AlumnoA}{Carlos Mira López}
\newcommand{\AlumnoB}{Nicolás Miró Mira}
\newcommand{\AlumnoC}{Vittorio Alessandro Esposito Ceballos}

\newcommand{\AlumnosPie}{\AlumnoA\ -- \AlumnoB\ -- \AlumnoC}
\newcommand{\Asignatura}{Asignatura}
\newcommand{\CursoTitulacion}{4$.\!^\circ$ curso - Grado en Ingeniería ...}
%https://www.rae.es/dpd/ordinales

\newcommand{\Data}{Octubre de 2022}

%------------------------------------------------------------------
% Configuració de formats i bibliografia

\input{./configuraciones/preambulo}
\input{./configuraciones/preambulo_listings}

% Si vols utilitzar un tipus de lletra semblant a Arial, descomenta les dos línies següents:
% \usepackage{cmbright}
% \usepackage[OT1]{fontenc}

\bibliography{./configuraciones/referencias}

%------------------------------------------------------------------
% Logo:

\setboolean{LogoUPV}{false}
\setboolean{LogoAlcoi}{true}

%------------------------------------------------------------------
%------------------------------------------------------------------
%------------------------------------------------------------------

\begin{document}

% -------------------------------------
% -------------------------------------

\input{./configuraciones/post_begin_document} % No eliminar!!!

% -------------------------------------
% -------------------------------------


%------------------------------------------------------------------
%------------------------------------------------------------------
% Resumen

%------------------------------------------------------------------
%------------------------------------------------------------------

\section{Introducción}
\label{sec:introduccion}
En este proyecto se diseña y construye un robot manipulador de al menos tres grados de libertad, controlado con servomotores y una placa Arduino UNO. El objetivo es que el robot pueda mover sus articulaciones de manera precisa, calcular la posición del efector final mediante cinemática directa, y determinar los ángulos necesarios para llegar a posiciones deseadas mediante cinemática inversa.
Además, se desarrolla una interfaz de usuario que permite controlar el robot, ver su posición en coordenadas articulares y cartesianas, y enviar comandos de movimiento de forma sencilla.

%------------------------------------------------------------------
%------------------------------------------------------------------

\section{Descripción del diseño mecánico y electrónico}
\label{sec:diseño}
Hemos optado por el diseño y ensamblaje de un robot angular con 3 grados de libertad, los cuales aparecen como una rotación en la base en el eje x, un movimiento horizontal en el eje z con respecto a la base gracias a un rodamiento que hemos implementado y un movimiento en el eje x del codo del robot, también gracias a un rodamiento implementado.
\\ \\
Utilizamos 3 servos, uno bajo la base para realizar el giro, otro alineado con el rodamiento del brazo para llevar a cabo el movimiento y otro alineado con el rodamiento del codo, también para su movimiento. En cuanto al cableado


%------------------------------------------------------------------
%------------------------------------------------------------------

\section{Modelo matemático}
\label{sec:modelo}


\subsection{Cinemática directa}

\subsection{Cinemática inversa}


%------------------------------------------------------------------
%------------------------------------------------------------------

\section{Código e interfaz}
\label{sec:código}



%------------------------------------------------------------------
\subsection{Código implementado en Arduino}


%------------------------------------------------------------------
\subsection{Interfaz de usuario}
\begin{figure}[H]
    \centering
    \includegraphics[width=0.65\linewidth]{figuras/interfaz.jpeg}
    \caption{}
\end{figure}

\section{Pruebas y demostración}
\label{sec:pruebas}
%------------------------------------------------------------------
%------------------------------------------------------------------

\bibitemsep = 2ex
\bibhang = 2em


%------------------------------------------------------------------
%------------------------------------------------------------------
%------------------------------------------------------------------

\end{document}

%------------------------------------------------------------------
%------------------------------------------------------------------
%------------------------------------------------------------------

