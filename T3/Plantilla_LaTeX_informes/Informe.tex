\documentclass[11pt]{article}

%------------------------------------------------------------------
%------------------------------------------------------------------
%------------------------------------------------------------------
% Informació de l'informe

\newcommand{\titol}{
	 Control por realimentación visual
	 }

\newcommand{\titolcap}{Control por realimentación visual}

\newcommand{\AlumnoA}{Carlos Mira López}
\newcommand{\AlumnoB}{Nicolás Miró Mira}
\newcommand{\AlumnoC}{Vittorio Alessandro Esposito Ceballos}

\newcommand{\AlumnosPie}{\AlumnoA\ -- \AlumnoB\ -- \AlumnoC}
\newcommand{\Asignatura}{Modelado y Control de Robots}
\newcommand{\CursoTitulacion}{4$.\!^\circ$ curso - Grado en Ingeniería ...}
%https://www.rae.es/dpd/ordinales

\newcommand{\Data}{Diciembre de 2025}

%------------------------------------------------------------------
% Configuració de formats i bibliografia

\input{./configuraciones/preambulo}
\input{./configuraciones/preambulo_listings}

% Si vols utilitzar un tipus de lletra semblant a Arial, descomenta les dos línies següents:
% \usepackage{cmbright}
% \usepackage[OT1]{fontenc}

\bibliography{./configuraciones/referencias}

%------------------------------------------------------------------
% Logo:

\setboolean{LogoUPV}{false}
\setboolean{LogoAlcoi}{true}

%------------------------------------------------------------------
%------------------------------------------------------------------
%------------------------------------------------------------------

\begin{document}

% -------------------------------------
% -------------------------------------

\input{./configuraciones/post_begin_document} % No eliminar!!!

% -------------------------------------
% -------------------------------------


%------------------------------------------------------------------
%------------------------------------------------------------------
% Resumen

%------------------------------------------------------------------
%------------------------------------------------------------------

\section{Introducción}
\label{sec:introduccion}
En este trabajo se desarrolla un sistema de control visual para el robot manipulador de 3 
GDL construido en el Trabajo 1. El propósito es que el robot pueda percibir la posición de 
un objeto mediante la cámara OV7670 y ajustar su movimiento en función de esa información. 
Para ello, se integra la cámara en el hardware del robot, se calibra para obtener coordenadas 
fiables y se implementa un algoritmo capaz de detectar y seguir un objeto de referencia dentro 
de la imagen. A partir de esta detección, se calcula el error visual entre la posición actual 
del objeto y la deseada, y dicho error se utiliza como señal de control para modificar la postura 
del robot en tiempo real. Con todo ello, se pretende demostrar que el manipulador es capaz de mantener 
la alineación respecto al objetivo y reaccionar de forma autónoma ante cambios en su posición dentro 
del campo de visión.
%------------------------------------------------------------------
%------------------------------------------------------------------

\section{Integración hardware de la cámara}
\label{sec:integracion}







%------------------------------------------------------------------
%------------------------------------------------------------------


\section{Algoritmo de procesamiento de imágenes}
\label{sec:algoritmo}




%------------------------------------------------------------------
%------------------------------------------------------------------

\section{Esquema de control visual}
\label{sec:control}




%------------------------------------------------------------------
%------------------------------------------------------------------
\section{Resultados experimentales}
\label{sec:resultados}


%------------------------------------------------------------------
%------------------------------------------------------------------


\section{Limitaciones y mejoras}
\label{sec:limitaciones}

%------------------------------------------------------------------
%------------------------------------------------------------------

\section{Demostración práctica}
\label{sec:demostración}


%------------------------------------------------------------------
%------------------------------------------------------------------


\bibitemsep = 2ex
\bibhang = 2em


%------------------------------------------------------------------
%------------------------------------------------------------------
%------------------------------------------------------------------

\end{document}

%------------------------------------------------------------------
%------------------------------------------------------------------
%------------------------------------------------------------------

