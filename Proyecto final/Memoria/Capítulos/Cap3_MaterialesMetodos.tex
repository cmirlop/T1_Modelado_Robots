% ----------------------------------------------------------------------------------
% ----------------------------------------------------------------------------------
% ----------------------------------------------------------------------------------
% ----------------------------------------------------------------------------------
\chapter{Materiales y métodos}\label{cap:Sol}

\begin{Resumen}

La sección Materials and Methods (también llamada Methodology o Experimental Section, según la disciplina) es una parte esencial de artículos y memorias de ámbito académico/docente. Su objetivo principal es que otro estudiante, profesor, investigador, ingeniero, etc., pueda reproducir el trabajo siguiendo las descripciones dadas.

\end{Resumen}

\section{Modelado cinemático/dinámico del robot + tool (si aplica)}
Descripción de los pasos seguidos para modelar el robot dentro del entorno de simulación. Si se ha realizado un modelado teórico, incorporar: Formulación de la cinemática directa e inversa mediante el uso de matrices de transformación homogénea, parámetros de Denavit–Hartenberg. Obtención de la matriz Jacobiana y análisis de singularidades.

Modelado dinámico (ecuaciones de Euler–Lagrange o Newton–Euler) con las hipótesis adoptadas (ej. robot rígido, sin rozamiento, etc.).

Representación clara de ecuaciones y, cuando sea útil, diagramas que apoyen la comprensión.
    
\section{Metodología de control y aprendizaje por refuerzo (RL)}
Explicación de los esquemas de control diseñados: control clásico de referencia, control basado en RL, comparación, etc.
\subsection{Esquema de control}
\subsubsection{Control por LIDAR}

\subsubsection{Control por RL}


Justificación de la elección de algoritmos de RL (ej.: PPO, SAC, DDPG), con breve descripción de su funcionamiento.
\subsection{Algoritmo RL PPO}



Definición de recompensas, estados y acciones empleados en el entorno de simulación.

\subsection{Recompensas, estados y acciones}
\subsubsection{recompensas}
Las recompensas que se le han asignado al robot se basán en si se ha descubierto area o no dentro de la imagen de 
la cámara. Se han definido zonas de distancia de LIDAR al robot para darle más recompensa o menos:
\begin{itemize}
    \item Distancia frontal libre(FRONT\_CLEAR) : cuando el lidar por delante detecta más de 10m
    \item Distancia frontal peligrosa(FRONT\_DANGER) : Cuando la distancia frontal del LIDAR por delante baja de los 7m
    \item Distancia muy peligrosa : Es la distancia antes del choque donde antes de chocarse prefiero que vaya hacia atrás el robot, 5.5m
\end{itemize}
Primero se comprueban las recompensas por alcanzar el objetivo, o por colisión:
\begin{itemize}
    \item Si se alcanza el objetivo : $+30.0$
    \item Si se ha colisionado : $-8.0$
\end{itemize}
Si no se ha alcanzado el área mínima de 1000px, las recompensas son las siguientes:
\begin{itemize}
    \item Si delante es muy libre, es decir, por encima del umbral FRONT\_CLEAR, y el robot tiene la acción de ir hacia delante rapido o lento le damos $0.3$ por rápido, y $0.15$ por lento
    \item Si delante esta por debajo del umbral FRONT\_DANGER y el robot tiene la accion de ir hacia delante, le quitamos $0.6$
    \item Si delante está por debajo del FRONT\_CLEAR, y giramos hacia el lado que más depejado esta, le damos $0.4$, con esto lo que hacemos es que si por delante vemos una distancia donde no cabe el robot, es inecesario entrar
    dentro de esa sala, es mejor girar
\end{itemize}

\subsubsection{Estados}

\subsubsection{Acciones}
\begin{itemize}
    \item Moverse hacia delante rápido
    \item Moverse hacia delante lento
    \item Moverse fuerte hacia la derecha
    \item Moverse suave hacia la derecha
    \item Moverse fuerte hacia la izquierda
    \item Moverse suave hacia la izquierda
    \item Moverle lento hacia antrás
\end{itemize}

\section{Entorno de simulación y herramientas utilizadas}
Descripción de IsaacSim/IsaacLab: versiones empleadas, configuración inicial y librerías auxiliares. Recursos computacionales (hardware, GPU, sistema operativo).

Configuración de escenarios de entrenamiento (robot, entorno, sensores virtuales, condiciones de interacción).

\subsection{Unity + Visual estudio}
%ros-TCP-connector como libreria

\subsection{Visual estudio code}

\subsection{TensorFlow}
TensorFlow Dashboard es una herramienta que a través de los logs que genera nuestro entramiento, va graficando 
cada 4024 pasos, que son 256 por robot, se grafican los resultados.



\subsection{KERAS + gym}



\subsection{ROS 1}
Hemos empleado ROS1 como pasarela de comunicacion entre los robots y el master, que seria nuestro propio ordenador. Lo que hacemos es 
que cada robot tiene unos topics propios, y sobre los cuales se envia y recibe la información




\subsection{Turtle bot}



\subsection{Entorno}



\subsection{Sensores virtuales}
\subsubsection{LIDAR}
Hemos modelado un LIDAR que publica los valores en un topic de ros, es este colisionado
es un codigo de Unity el cual genera una nube de puntos, y los publica todos a través del tópic

\subsection{Condiciones de iteraccion}



\section{Procedimiento de experimentación / entrenamiento}
Explicación del flujo de trabajo seguido: preparación de modelos, definición de hiperparámetros, duración de entrenamientos, validación de políticas aprendidas.
\subsection{Flujo de trabajo}
Se probo primero definiendo un escenario muy sencillo, donde si el robot giraba en el primer cruze ya alcanzaba
el objetivo final. Esto se modificó y se hicieron 4 escenarios para el mismo entrenamiento, donde el objetivo estaba 
en diferentes puntos.

Estrategia de comparación: métricas definidas (tiempo de convergencia, estabilidad, error en el seguimiento, etc.).
\subsection{Estrategia de comparacion}
Se definio que cada 1000 pasos en cada robot, se realize una comparcion de modelo, pero se detendria
el entrenamineto si la media de los ultimos 1000 episodios no mejoraba durante los 5 siguientes episodios.


Número de episodios, pruebas o repeticiones realizadas.

\section{Programación}
Diagramas de flujo o pseudocódigo de los programas implementados.

%Incluye todas las secciones y subsecciones que tu proyecto necesite para poder entender el trabajo que has realizado