% ----------------------------------------------------------------------------------
% ----------------------------------------------------------------------------------
% ----------------------------------------------------------------------------------
% ----------------------------------------------------------------------------------
\chapter{Materiales y métodos}\label{cap:Sol}

\begin{Resumen}

La sección Materials and Methods (también llamada Methodology o Experimental Section, según la disciplina) es una parte esencial de artículos y memorias de ámbito académico/docente. Su objetivo principal es que otro estudiante, profesor, investigador, ingeniero, etc., pueda reproducir el trabajo siguiendo las descripciones dadas.

\end{Resumen}

\section{Modelado cinemático/dinámico del robot + tool (si aplica)}
Descripción de los pasos seguidos para modelar el robot dentro del entorno de simulación. Si se ha realizado un modelado teórico, incorporar: Formulación de la cinemática directa e inversa mediante el uso de matrices de transformación homogénea, parámetros de Denavit–Hartenberg. Obtención de la matriz Jacobiana y análisis de singularidades.

Modelado dinámico (ecuaciones de Euler–Lagrange o Newton–Euler) con las hipótesis adoptadas (ej. robot rígido, sin rozamiento, etc.).

Representación clara de ecuaciones y, cuando sea útil, diagramas que apoyen la comprensión.
    
\section{Metodología de control y aprendizaje por refuerzo (RL)}
Explicación de los esquemas de control diseñados: control clásico de referencia, control basado en RL, comparación, etc.

Justificación de la elección de algoritmos de RL (ej.: PPO, SAC, DDPG), con breve descripción de su funcionamiento.

Definición de recompensas, estados y acciones empleados en el entorno de simulación.

\subsection{Recompensas, estados y acciones}
\subsubsection{recompensas}
Las recompensas que se le han as

\subsubsection{Estados}

\subsubsection{Acciones}

\section{Entorno de simulación y herramientas utilizadas}
Descripción de IsaacSim/IsaacLab: versiones empleadas, configuración inicial y librerías auxiliares. Recursos computacionales (hardware, GPU, sistema operativo).

Configuración de escenarios de entrenamiento (robot, entorno, sensores virtuales, condiciones de interacción).

\subsubsection{Unity + Visual estudio}

\subsubsection{Visual estudio code}

\subsubsection{ROS 1}

\subsubsection{robot}


\subsubsection{Entorno}


\subsubsection{Sensores virtuales}


\subsubsection{Condiciones de iteraccion}



\section{Procedimiento de experimentación / entrenamiento}
Explicación del flujo de trabajo seguido: preparación de modelos, definición de hiperparámetros, duración de entrenamientos, validación de políticas aprendidas.

Estrategia de comparación: métricas definidas (tiempo de convergencia, estabilidad, error en el seguimiento, etc.).

Número de episodios, pruebas o repeticiones realizadas.

\section{Programación}
Diagramas de flujo o pseudocódigo de los programas implementados.

%Incluye todas las secciones y subsecciones que tu proyecto necesite para poder entender el trabajo que has realizado