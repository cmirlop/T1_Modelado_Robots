% ----------------------------------------------------------------------------------
% ----------------------------------------------------------------------------------

\chapter{Introducción, objetivos y estructura del documento}\label{cap:Introducción}

\begin{Resumen}
Este capítulo debe situar al lector en el contexto general del proyecto y dejar claras las metas que se persiguen. Sirve como punto de partida para entender qué se va a hacer, por qué es relevante y qué se pretende conseguir.
\end{Resumen}

\section{Introducción}
Presentación del tema: la importancia de la robótica en la industria, la investigación y la vida cotidiana.

Justificación del proyecto: necesidad de comprender los fundamentos de cinemática, dinámica y control de robots para abordar retos actuales.

Relevancia del uso de técnicas avanzadas como el aprendizaje por refuerzo en entornos de simulación realistas (IsaacLab/IsaacSim).

Relación con el ámbito docente: adquisición de competencias en modelado matemático, simulación y control inteligente.

\section{Objetivo general y objetivos específicos}
Una frase clara que resuma la meta principal del proyecto.

Listado numerado y concreto de metas parciales que permiten alcanzar el objetivo general.

\section{Estructura del documento}
Breve resumen del contenido de cada uno de los capítulos que consta la memoria.

    