% ----------------------------------------------------------------------------------
% ----------------------------------------------------------------------------------

\chapter{Introducción, objetivos y estructura del documento}\label{cap:Introducción}

\begin{Resumen}
El presente proyecto aborda el desafío de la navegación autónoma en robótica móvil mediante 
el uso de percepción visual. Se centra en el modelado y control del robot TurtleBot3 dentro 
de un entorno de simulación de alta fidelidad (Unity). El objetivo fundamental es 
desarrollar estrategias de control que permitan al robot alcanzar objetivos definidos utilizando 
cámaras RGB/RGB-D, explorando tanto algoritmos de navegación clásicos como técnicas avanzadas de 
Aprendizaje por Refuerzo (Reinforcement Learning), evaluando su desempeño en escenarios controlados.
\end{Resumen}

\section{Introducción}

La robótica móvil ha experimentado una transformación radical en la última década, 
pasando de ser una curiosidad académica a convertirse en un pilar fundamental de 
la Industria 4.0 y la logística moderna. Los robots móviles autónomos (AMR) son 
hoy esenciales en tareas de transporte de materiales, inspección de infraestructuras 
y robótica de servicios. Sin embargo, para que estos sistemas sean verdaderamente autónomos
, deben poseer la capacidad de percibir su entorno, interpretarlo y tomar decisiones de navegación seguras y eficientes en tiempo real.

Tradicionalmente, la navegación se ha resuelto mediante mapas estáticos y sensores 
láser (LiDAR). No obstante, la tendencia actual se dirige hacia el uso de percepción 
visual (cámaras RGB y de profundidad) y técnicas de inteligencia artificial, que permiten 
obtener una información semántica del entorno mucho mejor y a un menor coste de hardware.

Este proyecto surge de la necesidad de comprender y dominar las técnicas modernas de control 
robótico. Específicamente, se justifica en la transición tecnológica actual que busca sustituir 
o complementar la programación clásica con algoritmos de aprendizaje. El uso de herramientas de 
simulación fotorrealista y física precisa, como NVIDIA IsaacSim e IsaacLab, permite entrenar 
y validar estos sistemas en entornos virtuales seguros antes de su despliegue físico, reduciendo riesgos 
y costes (concepto conocido como \textit{Sim-to-Real}).

Desde una perspectiva docente y académica, este trabajo permite la adquisición de competencias transversales 
en modelado cinemático, visión por computador, integración de sensores y, fundamentalmente, en la aplicación 
de Deep Reinforcement Learning (Deep RL), una técnica que está redefiniendo el estado del arte en el control de robots dinámicos.

\section{Objetivo general y objetivos específicos}

El \textbf{objetivo principal} de este proyecto es diseñar, implementar y validar un 
sistema de control para la navegación autónoma del robot móvil TurtleBot3 en un entorno 
simulado, capaz de localizar y alcanzar objetivos visuales utilizando información sensorial 
proveniente de cámaras a bordo y algoritmos de inteligencia artificial.

Para alcanzar esta meta, se han definido los siguientes objetivos específicos:

\begin{enumerate}
    \item \textbf{Modelado y configuración del entorno de simulación:} Importar y configurar 
    el modelo cinemático y visual del TurtleBot3 en el entorno Unity, integrando 
    sensores virtuales de visión (cámara RGB/RGB-D) y configurando escenarios con metas visuales específicas.
    
    \item \textbf{Desarrollo del sistema de percepción y navegación:} Implementar los 
    algoritmos necesarios para procesar la información visual del robot y traducirla en 
    comandos de velocidad (lineal y angular) que permitan la navegación hacia el objetivo.
    
    \item \textbf{Implementación de agentes de Aprendizaje por Refuerzo (RL):} Diseñar y 
    entrenar un agente basado en RL (explorando arquitecturas con redes neuronales) capaz de aprender la política de navegación óptima mediante la interacción prueba-error con el entorno simulado.
    
    \item \textbf{Validación y análisis comparativo:} Evaluar el rendimiento del sistema 
    en términos de tasa de éxito, tiempo de llegada y suavidad de la trayectoria, comparando 
    la robustez de las soluciones frente a cambios en la posición del objetivo o del entorno.
    
    \item \textbf{Documentación y análisis crítico:} Analizar la viabilidad de la transferencia 
    de estos algoritmos a robots de servicio reales y documentar el proceso técnico para futuras investigaciones.
\end{enumerate}

\section{Estructura del documento}

La presente memoria se organiza en cinco capítulos que detallan el desarrollo del proyecto desde su concepción 
teórica hasta la validación de resultados:

\begin{itemize}
    \item \textbf{Capítulo 1: Introducción, objetivos y estructura.} Introduce el contexto del problema, 
    justifica la elección del TurtleBot3 y las herramientas de simulación, y define las metas del proyecto.
    
    \item \textbf{Capítulo 2: Antecedentes y estado del arte.} Revisa los fundamentos teóricos de la robótica 
    móvil diferencial, la percepción visual y el Aprendizaje por Refuerzo. Se analizan trabajos previos y se 
    describen las herramientas utilizadas (ROS2, IsaacSim, PyTorch).
    
    \item \textbf{Capítulo 3: Materiales y métodos.} Detalla la metodología empleada, incluyendo el modelado 
    matemático del robot, la configuración de los sensores en simulación, la arquitectura de las redes neuronales 
    utilizadas y el diseño de la función de recompensa para el entrenamiento del agente.
    
    \item \textbf{Capítulo 4: Resultados.} Presenta las gráficas de entrenamiento, las métricas de desempeño 
    obtenidas en las pruebas de navegación y una comparativa de los distintos controladores implementados. Se 
    valida el cumplimiento de los objetivos.
    
    \item \textbf{Capítulo 5: Conclusiones y trabajo futuro.} Sintetiza los hallazgos principales, discute 
    las limitaciones encontradas durante el desarrollo y propone líneas de investigación para continuar el 
    trabajo, como la implementación en hardware real.
\end{itemize}
    