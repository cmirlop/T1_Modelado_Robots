% ----------------------------------------------------------------------------------
% ----------------------------------------------------------------------------------

\chapter{Introducción, objetivos y estructura del documento}\label{cap:Introducción}

\begin{Resumen}

\end{Resumen}

\section{Introducción}

La robótica móvil ha evolucionado significativamente en los últimos años, consolidándose como 
un elemento clave en la Industria 4.0 y la logística. Los robots móviles autónomos (AMR) son 
esenciales para tareas de transporte, inspección y servicios. Sin embargo, para que estos sistemas 
operen de forma efectiva, requieren la capacidad de percibir su entorno y tomar decisiones de navegación 
en tiempo real. Tradicionalmente, la navegación se ha basado en mapas estáticos y sensores láser (LiDAR). 
No obstante, existe una tendencia creciente hacia el uso de percepción visual y técnicas de inteligencia 
artificial, lo cual permite obtener información semántica del entorno reduciendo la dependencia de sensores costosos.

Este proyecto surge de la necesidad de aplicar técnicas modernas de control robótico. Se justifica en el interés 
actual por complementar la programación clásica con algoritmos de aprendizaje. Para ello, se opta por el uso de 
Unity como motor de simulación. Unity ofrece un entorno físico robusto y flexible que permite entrenar y validar 
estos sistemas de forma segura antes de una eventual implementación física, reduciendo riesgos y costes asociados a las 
pruebas en hardware real.

Desde una perspectiva académica, este trabajo facilita la adquisición de competencias en modelado cinemático, visión por 
computador y la aplicación de Deep Reinforcement Learning (Deep RL), permitiendo estudiar el comportamiento de agentes 
autónomos en entornos virtuales complejos.

\section{Objetivo general y objetivos específicos}

El \textbf{objetivo principal} es diseñar, implementar y validar un sistema de control para la navegación autónoma del TurtleBot3 
en un entorno simulado en Unity, capaz de localizar y alcanzar objetivos visuales utilizando información de cámaras a bordo.

Para alcanzar esta meta, se definen los siguientes objetivos específicos:

\begin{enumerate}
    \item \textbf{Configuración del entorno de simulación:} Integrar el modelo cinemático y visual del TurtleBot3 en el motor Unity, incorporando sensores virtuales de visión y laser(LiDAR) y diseñando escenarios de prueba con metas específicas.
    
    \item \textbf{Desarrollo del sistema de navegación:} Implementar los algoritmos de control necesarios para procesar la información visual y generar los comandos de velocidad (lineal y angular) para el desplazamiento del robot.
        
    \item \textbf{Validación y análisis:} Evaluar el rendimiento del sistema midiendo la tasa de éxito y la eficiencia de las trayectorias, comparando el comportamiento del robot ante diferentes configuraciones del entorno.
    
    \item \textbf{Documentación:} Registrar el proceso técnico, los problemas encontrados y las soluciones adoptadas, analizando la viabilidad de los algoritmos propuestos.
\end{enumerate}

\section{Estructura del documento}

La memoria se organiza en cinco capítulos que cubren desde la teoría hasta los resultados experimentales:

\begin{itemize}
    \item \textbf{Capítulo 1: Introducción, objetivos y estructura.} Presenta el contexto del problema, justifica el uso de Unity y el TurtleBot3, y define las metas del trabajo.
    
    \item \textbf{Capítulo 2: Antecedentes y estado del arte.} Revisa los fundamentos de la robótica diferencial y el Aprendizaje por Refuerzo. Se describen las herramientas clave utilizadas.
    
    \item \textbf{Capítulo 3: Materiales y métodos.} Detalla la metodología, incluyendo el modelado del robot en Unity, la configuración de los sensores, la arquitectura de control y el diseño de la función de recompensa.
    
    \item \textbf{Capítulo 4: Resultados.} Muestra las métricas de desempeño obtenidas en las pruebas de navegación, gráficas de entrenamiento y una discusión sobre el comportamiento del robot.
    
    \item \textbf{Capítulo 5: Conclusiones y trabajo futuro.} Sintetiza los logros del proyecto, discute las limitaciones detectadas y propone posibles mejoras o líneas de continuación.
\end{itemize}