% ----------------------------------------------------------------------------------
% ----------------------------------------------------------------------------------
% ----------------------------------------------------------------------------------
% ----------------------------------------------------------------------------------
\chapter{Antecedentes y estado del arte}\label{cap:Antecedentes}

\begin{Resumen}

En este capítulo se debe contextualizar el proyecto dentro del conocimiento existente. No se trata de hacer un simple resumen, sino de demostrar que se entiende qué se ha hecho ya en el área, qué problemas siguen abiertos y qué aporta vuestro trabajo en ese contexto.

\end{Resumen}

\section{Fundamentos teóricos relevantes}
Breve repaso de los conceptos esenciales de cinemática y dinámica de robots.

Introducción a los esquemas de control clásicos (PID, control por realimentación, control óptimo, etc.).

Principios básicos del aprendizaje por refuerzo (Markov Decision Processes, política, recompensa, exploración vs. explotación).

\section{Estado del arte en robótica y control}
Aplicaciones recientes de cinemática y dinámica en robots manipuladores y móviles.

Técnicas actuales de control: comparación entre enfoques tradicionales y métodos basados en inteligencia artificial.

Uso del aprendizaje por refuerzo en robótica: principales algoritmos (DQN, PPO, SAC, DDPG) y retos en su implementación.

\section{Herramientas y entornos de simulación}
Breve revisión de los simuladores más utilizados en robótica (Gazebo, MuJoCo, PyBullet, IsaacSim).

Justificación del uso de IsaacLab/IsaacSim para este proyecto.

\section{Trabajos y resultados previos}
Ejemplos de investigaciones o proyectos similares que hayan aplicado RL en robótica.

Limitaciones encontradas en esos trabajos y vacíos de investigación que motivan este proyecto.

\section{Relación con el proyecto}
Identificación de qué aspectos se tomarán como base para el trabajo (ej. modelado cinemático/dinámico tradicional).

Qué parte supone un reto o innovación (ej. implementación de RL en IsaacLab).
%Tantas secciones y subsecciones como te sea necesario


















