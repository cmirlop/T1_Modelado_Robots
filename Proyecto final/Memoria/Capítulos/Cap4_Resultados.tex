% ----------------------------------------------------------------------------------
% ----------------------------------------------------------------------------------
%------------------------------------------------------------------
% ----------------------------------------------------------------------------------
\chapter{Resultados}
\label{sec:Resultados}

\begin{Resumen}
En este capítulo se deben presentar y analizar los resultados obtenidos tras la implementación del proyecto. No se trata solo de mostrar datos, gráficos o tablas, sino de interpretarlos y relacionarlos con los objetivos planteados.

\end{Resumen}

\section{Validación del modelado matemático}
Comparación entre los resultados teóricos (cinemática y dinámica calculadas) y los obtenidos en simulación.

Comprobación de trayectorias, posiciones finales y detección de posibles singularidades.

Discusión sobre la precisión y las limitaciones del modelo.

\section{Entrenamientos en Unity}
Presentación de las curvas de aprendizaje (recompensa acumulada, convergencia del algoritmo).

Evaluación de los tiempos de entrenamiento y de la estabilidad de la política aprendida.

Análisis de hiperparámetros: cómo afectan al rendimiento y a la velocidad de convergencia.

\section{Comparación de controladores}
\subsection{Comparación control clásico vs control basado en RL}
%Resultados de control clásico vs. control basado en RL.
PONER VIDEOOOOO
Como observamos en el siguiente video :   podemos observar en una pantalla el comportamiento del robot en 
cuatro escenarios moviendose con el LIDAR, mientras que en la derecha vemos el robot intentando alcanzar
un objetivo después de realizar un entrenamiento. Podemos observar que con el LIDAR no produce colisiones, ya que
es muy estricto con toda la información del LIDAR que tiene, manteniendo la distancia con la pared izquierda 
sin problemas, mientras que con el comportamiento en RL vemos que se queda atascado en un giro, donde no mantiene 
la distancia con el lIDAR y se colisona.




Métricas de desempeño: error en el seguimiento de trayectorias, suavidad de los movimientos, robustez frente a perturbaciones.

Discusión sobre ventajas y limitaciones de cada enfoque.
\subsection{Ventajas y limitaciones}
Las ventajas del control por LIDAR, es que resuelve el escenario correctamente, ya que siempre irá apegado a la pared 
izquierda, pero esto tiene un gran inconveniente y esque podria llegar a escanear todo el entorno  si el objetivo esta a la derecha y oculto,
esto a difencia del RL es un problema, ya que con el RL se intenta aprender un patrón para no tener que ir siempre
apegado a la izquierda y poder hacer los movimientos más centrados.

\section{Pruebas finales y validación del sistema}

Ejemplos de simulaciones finales con el robot ejecutando las tareas propuestas.

Posibles animaciones, capturas de pantalla o gráficas que ilustren los comportamientos logrados.

Evaluación cualitativa (ej.: naturalidad de movimientos, respuesta a cambios en el entorno).

\section{Discusión crítica}
Relación de los resultados con los objetivos generales y específicos planteados en la introducción.
Respecto a los objetivos que propusimos en la introducción, hemos conseguido alcanzar todos los específicos,
ya que estos eran el control de un robot con camara y sensorica por un entorno para alcanzar el objetivo, esto se ha conseguido ya 
que como hemos visto anteriormente, el robot consigue alcanzar el objetivo con la cámara y el LIDAR, mientras que respecto 
a los específicos, hemos intentado el del reinforce learning, consiguiendo entrenar el robot, y que el robot se
desplazara por el laberinto aprendiendo la politica del robot con el LIDAR de seguir la paret izquierda, pero 
se colisionaba alfinal sin conseguir alcanzar el objetivo final.


Identificación de fortalezas, limitaciones y posibles mejoras del trabajo.

%Puedes poner todas las secciones y subsecciones que necesites para demostrar el funcionamiento de tu propuesta