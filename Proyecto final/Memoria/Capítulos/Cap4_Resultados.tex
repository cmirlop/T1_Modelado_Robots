% ----------------------------------------------------------------------------------
% ----------------------------------------------------------------------------------
%------------------------------------------------------------------
% ----------------------------------------------------------------------------------
\chapter{Resultados}
\label{sec:Resultados}

\begin{Resumen}
En este capítulo se deben presentar y analizar los resultados obtenidos tras la implementación del proyecto. No se trata solo de mostrar datos, gráficos o tablas, sino de interpretarlos y relacionarlos con los objetivos planteados.

\end{Resumen}

\section{Validación del modelado matemático}
Comparación entre los resultados teóricos (cinemática y dinámica calculadas) y los obtenidos en simulación.

Comprobación de trayectorias, posiciones finales y detección de posibles singularidades.

Discusión sobre la precisión y las limitaciones del modelo.

\section{Entrenamientos en IsaacLab/IsaacSim}
Presentación de las curvas de aprendizaje (recompensa acumulada, convergencia del algoritmo).

Evaluación de los tiempos de entrenamiento y de la estabilidad de la política aprendida.

Análisis de hiperparámetros: cómo afectan al rendimiento y a la velocidad de convergencia.

\section{Comparación de controladores}
Resultados de control clásico vs. control basado en RL.

Métricas de desempeño: error en el seguimiento de trayectorias, suavidad de los movimientos, robustez frente a perturbaciones.

Discusión sobre ventajas y limitaciones de cada enfoque.

\section{Pruebas finales y validación del sistema}

Ejemplos de simulaciones finales con el robot ejecutando las tareas propuestas.

Posibles animaciones, capturas de pantalla o gráficas que ilustren los comportamientos logrados.

Evaluación cualitativa (ej.: naturalidad de movimientos, respuesta a cambios en el entorno).

\section{Discusión crítica}
Relación de los resultados con los objetivos generales y específicos planteados en la introducción.

Identificación de fortalezas, limitaciones y posibles mejoras del trabajo.

%Puedes poner todas las secciones y subsecciones que necesites para demostrar el funcionamiento de tu propuesta